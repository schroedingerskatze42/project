\section{Frameworks}

Im Umfeld des Machine Learnings sind zahlreiche Frameworks für verschiedene Plattformen und verschiedene Sprachen zu
finden. Diese sind auf unterschiedlichen Abstraktionsebenen angesiedelt und dementsprechend unterschiedlich komplex
in der Verwendung.
Je höher ein Framework dabei angesiedelt, desto einfacher ist es üblicherweise zu bedienen, bietet dem Anwender aber
dafür nur sehr stark eingeschränkte Möglichkeiten. Frameworks hingegen, die auf einem niedrigen Level operieren, bieten
dem Anwender sehr detaillierte und komplexe Möglichkeiten, den gesamtem Programmfluss zu beeinflussen, sind dadurch aber
auch deutlich komplexer zu verwenden.
Die Frameworks sind insbesondere darauf ausgerichtet, eine Implementation für die konkreten mathematischen Berechnungen
und Modelle zu bieten, die darüber hinaus auch hinsichtlich der Performance optimiert worden sind.

Zu nennen wären beispielsweise die nachfolgenden drei Frameworks:

\begin{enumerate}
    \item{Scikit-Learn\footnote{scikit-learn, Machine Learning in Python.\newline(http://scikit-learn.org/stable/)}}
    \item{Keras\footnote{Keras: The Python Deep Learning library.\newline(https://keras.io/)}}
    \item{TensorFlow\footnote{TensorFlow\texttrademark, An open-source machine learning framework for everyone.\newline(https://www.tensorflow.org/)}}
\end{enumerate}

Diese Aufzählung ist nicht vollständig und stellt lediglich einige Repräsentanten dar.

\subsection{Scikit-Learn}

Scikit-Learn stellt einen simplen Einstieg in die Welt des Machine Learnings dar und hält den Großteil der eigentlichen
Komplexität vor dem Anwender verborgen. Scikit-Learn baut im Wesentlichen auf NumPy [http://www.numpy.org/] und
SciPy [https://www.scipy.org/] auf und bietet eine Reihe von einfach zu verwendenden Werkzeugen, die für die Analyse und
Auswertung von Daten genutzt werden können.

\subsection{Keras}

Ähnlich wie Scikit-Learn stellt Keras ebenfalls ein Framework dar, das auf einer verhältnismäßig hohen Ebene operiert.
Die Besonderheit besteht dabei darin, dass Keras eine Schnittstelle zu komplexeren Frameworks wie beispielsweise
TensorFlow oder Theano bereitstellt. Damit eignet Keras sich insbesondere für das schnelle und einfache Entwickeln von
Prototypen.

\subsection{TensorFlow}

Von den beiden zuvor genannten Frameworks kann TensorFlow abgegrenzt werden. TensorFlow bietet eine Reihe von
verschieden komplexen APIs, die dem Anwender Zugang zu verschiedenen Leveln ermöglichen. Die Architektur des Frameworks
kann mit Hilfe der nachfolgenden Grafik verdeutlicht werden:

\begin{figure}[h]
    \centering
    \includegraphics[width=0.3\textwidth]{docu_img_19}
    \caption{XXXXXXXXX}
    \label{fig:foo}
\end{figure}
docuimg21

Wie Keras und Scikit-Learn bietet auch TensorFlow die Möglichkeit, Verfahren zu verwenden, die leicht zu bedienen sind
und somit keine tiefgehenden Kenntnisse voraussetzen. Darüber hinaus ist es allerdings auch möglich, eigene Modelle und
Strukturen von Grund auf selbst zu definieren und so direkten Einfluss auf das Lernen, Validieren und Vorhersehen zu
nehmen.

\subsubsection{Verwendung von TensorFlow}
