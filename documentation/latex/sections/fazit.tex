\section{Fazit}

Die Ergebnisse, die dieses Projekt liefert, sind von einem verwendbaren Resultat noch etwas entfernt. Dennoch sieht man eine
valide Approximation, die auch nach zehn Epochen noch nicht abgeschlossen zu sein scheint. Wir sind der Überzeugung, dass die Resultate,
mit Anpassungen folgender Parameter, durchaus brauchbar werden können.

\subsection{Trainingsdatensätze}
Ideal wäre eine Datenbank ähnlich der \textit{FaceScrub}\textsuperscript{\ref{footnote:face-scrub}} jedoch mit mehr
individuellen Personen in natürlicher Umgebung. Denn soll das Netz dazu verwendet werden, Bilder widerherzustellen, so
besteht dort kein Einfluss auf den Kontext, in dem das Ursprungsbild aufgenommen wurde.
Zudem sollte es sich als hilfreich erweisen, Bilder in einer möglichst großen Auflösung zum Trainieren zu verwenden, da
eine Vergrößerung der Urspungsbilder verlustfrei möglich ist. Muss das Ursprungsbild jedoch zuvor runterskaliert werden, geht
an dieser Stelle schon Information verloren, die sonst dazu dienen könnte, eine mögliche Ursprungsinformation wiederherzustellen.

\subsection{Lerndauer}
Das Netz noch mehr zu trainieren war, mit den uns zur Verfügung stehenden Computern, nicht möglich. Eine unterstüzende Maßnahme
hierfür könnte \textit{Cloud Computing}\footnote{Wikipedia, Cloud Computing.\newline(https://de.wikipedia.org/wiki/Cloud\_Computing)}
sein, mit der es möglich wäre noch mehr Datensätze in einer noch kürzeren Zeit zu lernen.
